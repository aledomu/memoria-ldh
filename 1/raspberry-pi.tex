Instalaremos la distribución Raspbian 10 Lite. Para ello, descargaremos la
imagen recomendada para la práctica\footnotemark. Luego, descomprimiremos el
archivo, lo cual nos dará una imagen que deberemos grabar en una tarjeta SD. En
Linux, esto puede realizarse mediante la siguiente orden desde la terminal
desde el directorio donde se encuentra la imagen, siendo \verb|/dev/sdb| el
nodo de dispositivo correspondiente a la tarjeta:

\footnotetext{\url{
    https://coria.dte.us.es/~bellido/2021-05-07-raspios-buster-armhf-lite.zip
}}

\begin{lstlisting}[language=sh]
sudo dd if=./2021-05-07-raspios-buster-armhf-lite.img of=/dev/sdb bs=4M status=progress
\end{lstlisting}

Podrían usarse otras distribuciones Linux, como Fedora IoT. Sin embargo, el
problema es que el software que instalaremos en ejercicios posteriores
requiere de una interfaz al puerto GPIO obsoleta que no todas las
distribuciones tienen habilitada\footnotemark.

\footnotetext{
    Esta incidencia ya ha sido reportada a los desarrolladores en cuanto ha
    sido descubierta: \url{https://github.com/mysensors/MySensors/issues/1536}
}

Para habilitar el acceso por SSH desde el primer arranque, colocaremos un
archivo vacío llamado \verb|ssh| (sin extensión) en la raíz de la partición
etiquetada como \verb|boot|. Sin embargo, para que esto sea efectivo sin más,
la conexión a la red deberá ser por cable y sin necesitar adquirir una
dirección IP estática desde el propio dispositivo. Sin embargo, no es este el
caso, dado que usaremos una red WiFi con dirección IP estática. Para configurar
la conexión a la red WiFi adecuada, en nuestro caso con SSID y contraseña
\verb|honeypot| con cifrado WPA personal (WPA-PSK), debemos poner el siguiente
archivo en la raíz de la partición \verb|boot|:

\lstinputlisting[caption=wpa\_supplicant.conf]{
    1/raspberry-pi/wpa_supplicant.conf
}

Para la asignación estática de IP, añadiremos el siguiente texto al final del
archivo \verb|/etc/dhcpcd.conf| en la partición etiquetada \verb|rootfs|:

\lstinputlisting{1/raspberry-pi/dhcpcd.conf}

No se ha usado una conexión cableada dado que ello requeriría el uso de dos
puestos en el aula. Con el método empleado, sin embargo, solo hace falta un
puesto, y en caso de usar un punto de acceso propio (por ejemplo, con el móvil)
ni siquiera se depende del setup del aula.

Al iniciar la Raspberry Pi con la tarjeta insertada, nos conectaremos por SSH
con el usuario \verb|pi| y la contraseña \verb|raspberry| a la dirección IP de
la Raspberry Pi, en nuestro caso \verb|10.42.0.90|. Una vez dentro,
ejecutaremos la herramienta de configuración con permisos de administrador:

\begin{lstlisting}[language=sh]
sudo raspi-config
\end{lstlisting}

Pondremos los siguientes ajustes:

\begin{itemize}
\item \emph{System Options}:
\begin{itemize}
\item \emph{Password}: (cualquiera)
\item \emph{Hostname}: rpi90
\end{itemize}
\item \emph{Interface Options $\rightarrow$ SPI}: Yes
\item \emph{Localisation Options}:
\begin{itemize}
\item \emph{Locale}: es\_ES.UTF-8
\item \emph{Timezone}: Europe, Madrid
\item \emph{WLAN Country}: ES Spain
\end{itemize}
\item \emph{Advanced Options $\rightarrow$ Expand Filesystem}
\end{itemize}

Una vez acabado, cerramos y reiniciamos, tal y como nos pide. Luego volveremos
a ejecutar la herramienta y generaremos el mapa de teclado adecuado yendo a
\emph{Localisation Options $\rightarrow$ Keyboard}.

Finalmente, actualizamos el sistema sin cambiar de versión para recibir mejoras
de seguridad en caso de que existan, reiniciando otra vez tras los cambios:
\begin{lstlisting}[language=sh]
sudo apt-get update -y
sudo apt-get upgrade -y
\end{lstlisting}